Tail queries have been often mapped to the head \cite{} by using bag-of-word representations.
Lately they have also been linked with the head queries through entities. However, in this work
we attempt to quantify the extent to which the entities in the tail can be linked to entities 
in the head queries. It is important to know how much will entity linking at tail benefit query mining. 
We shall, from now on refer to entities in head queries as `head entity' 
and entities occuring in the tail as `tail entity'. 

%1.no of queries with head and tail ent
%2.Head and tail entity histogram for queries with more than 1 ent 
We begin by analyzing the volume of entities found in the tail. With X unique tail queries, we 
found that 63.2\% contain atleast one entity. Of queries containing more than one entity,  
86.2\% queries contain atleast one head entity. Table \ref{table:entDist} shows the number of 
queries, corresponding number of entities and the head to tail entity ratio in query. 
Head to Tail entity ratio is calculated by dividing number of head entities with number of 
tail entities in the query.
\begin{table}
\caption{\#Entities vs \#Queries}
\label{table:entDist}
\centering
\begin{tabular}{|l|l|l\}
\hline
\#Entities & \#Queries & head\-tail Ratio \\ \hline
1 & 2622873 & na\\hline
2 & 1731256 & 0.48\\ \hline
3 & 457946  & 0.91\\ \hline
4 & 74125 & 1.09 \\ \hline
5 & 10757 & 1.07\\ \hline
6 & 2416  & 0.97\\ \hline
\end{tabular}
\end{table}

Figure \ref{img:headTailEntBreakup} shows the percentage of head and tail entities with number 
of entities in a query. Although, there are queries with more than 6 entities, 
they are too few to draw any significant conclusions. 

%
Given that there would several head entities, there is a pattern in the 


In entity analysis of Aol query logs we calculate the following:
\begin{itemize}
\item No of queries with entities from the head or tail
\item Frequency distribution of entities in head and tail
\item Head and tail entity distribution in queries
\ent{itemize}





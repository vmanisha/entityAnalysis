% THIS IS SIGPROC-SP.TEX - VERSION 3.1
% WORKS WITH V3.2SP OF ACM_PROC_ARTICLE-SP.CLS
% APRIL 2009
%
% It is an example file showing how to use the 'acm_proc_article-sp.cls' V3.2SP
% LaTeX2e document class file for Conference Proceedings submissions.
% ----------------------------------------------------------------------------------------------------------------
% This .tex file (and associated .cls V3.2SP) *DOES NOT* produce:
%       1) The Permission Statement
%       2) The Conference (location) Info information
%       3) The Copyright Line with ACM data
%       4) Page numbering
% ---------------------------------------------------------------------------------------------------------------
% It is an example which *does* use the .bib file (from which the .bbl file
% is produced).
% REMEMBER HOWEVER: After having produced the .bbl file,
% and prior to final submission,
% you need to 'insert'  your .bbl file into your source .tex file so as to provide
% ONE 'self-contained' source file.
%
% Questions regarding SIGS should be sent to
% Adrienne Griscti ---> griscti@acm.org
%
% Questions/suggestions regarding the guidelines, .tex and .cls files, etc. to
% Gerald Murray ---> murray@hq.acm.org
%
% For tracking purposes - this is V3.1SP - APRIL 2009

%\documentclass{style/acm_proc_article-sp}

\documentclass{style/sig-alternate-2013}[9pt]

\usepackage{color}
%\usepackage{pgfplotstable}
\usepackage{booktabs}
\usepackage{array}
\usepackage{colortbl}
%\usepackage{algorithm2e}
%\usepackage{algorithm}
\usepackage{url}
\usepackage{stmaryrd}
\usepackage{enumitem}
\usepackage{xspace}
\usepackage{multirow}
\usepackage[show]{style/chato-notes}
%\usepackage{pgfplots}

%\usepackage{tikz}
%\usetikzlibrary{positioning}

\pdfpagewidth=8.5in
\pdfpageheight=11in

\definecolor{mybluei}{RGB}{124,156,205}
\definecolor{myblueii}{RGB}{73,121,193}
\definecolor{mygreen}{RGB}{202,217,126}

%\pgfdeclarelayer{background}
%\pgfsetlayers{background,main}


\newcommand{\comment}[2]{[\textbf{\color{red} #1}: \emph{#2}]\\}%
% uncomment below to hide comments
%\renewcommand{\comment}[2]{}

\newcommand{\diego}[1]{\comment{diego}{#1}}
\newcommand{\claudio}[1]{\comment{claudio}{#1}}
\newcommand{\manisha}[1]{\comment{manisha}{#1}}


%\newcommand{\todo}[1]{\comment{todo}{#1}}


\newcommand{\tail}[1]{$Q_{tail}${}}
\newcommand{\head}[1]{$Q_{head}${}}
\newcommand{\stail}[1]{$S_{tail}${}}
\newcommand{\shead}[1]{$S_{head}${}}
\newcommand{\etail}[1]{$E_{tail}${}}
\newcommand{\ehead}[1]{$E_{head}${}}

\newcommand{\dexter}[1]{{\sf Dexter}\xspace}


\newcommand{\piccino}[1]{ {\footnotesize #1}}
\newcommand{\etal}[1]{ {\emph{et al.}\~{}}}
%\newcommand{\eg}[1]{ {\emph{e.g.,}\~{}}}


\newtheorem{example}{Example}[section]
\newtheorem{problem}{Problem}[section]
\newtheorem{proposition}{Proposition}[section]







\begin{document}
	
	
% --- Author Metadata here ---
\newfont{\mycrnotice}{ptmr8t at 7pt}
\newfont{\myconfname}{ptmri8t at 7pt}
\let\crnotice\mycrnotice%
\let\confname\myconfname%

\permission{Permission to make digital or hard copies of all or part of this work for personal or classroom use is granted without fee provided that copies are not made or distributed for profit or commercial advantage and that copies bear this notice and the full citation on the first page. Copyrights for components of this work owned by others than ACM must be honored. Abstracting with credit is permitted. To copy otherwise, or republish, to post on servers or to redistribute to lists, requires prior specific permission and/or a fee. Request permissions from permissions@acm.org.}
%\conferenceinfo{ESAIR'13,}{October 28, 2013, San Francisco, CA, USA.}
%\CopyrightYear{2013}
%\crdata{978-1-4503-2413-7/13/10 \\
%http://dx.doi.org/10.1145/2513204.2513212 }



\clubpenalty=10000 
\widowpenalty = 10000


\title{Bringing the Head Closer to the Tail with Entity Linking}
%\subtitle{[Extended Abstract]
%\titlenote{A full version of this paper is available as
%\textit{Author's Guide to Preparing ACM SIG Proceedings Using
%\LaTeX$2_\epsilon$\ and BibTeX} at
%\texttt{www.acm.org/eaddress.htm}}}
%
% You need the command \numberofauthors to handle the 'placement
% and alignment' of the authors beneath the title.
%
% For aesthetic reasons, we recommend 'three authors at a time'
% i.e. three 'name/affiliation blocks' be placed beneath the title.
%
% NOTE: You are NOT restricted in how many 'rows' of
% "name/affiliations" may appear. We just ask that you restrict
% the number of 'columns' to three.
%
% Because of the available 'opening page real-estate'
% we ask you to refrain from putting more than six authors
% (two rows with three columns) beneath the article title.
% More than six makes the first-page appear very cluttered indeed.
%
% Use the \alignauthor commands to handle the names
% and affiliations for an 'aesthetic maximum' of six authors.
% Add names, affiliations, addresses for
% the seventh etc. author(s) as the argument for the
% \additionalauthors command.
% These 'additional authors' will be output/set for you
% without further effort on your part as the last section in
% the body of your article BEFORE References or any Appendices.

\numberofauthors{3} %  in this sample file, there are a *total*
% of EIGHT authors. SIX appear on the 'first-page' (for formatting
% reasons) and the remaining two appear in the \additionalauthors section.
%
\author{
% You can go ahead and credit any number of authors here,
% e.g. one 'row of three' or two rows (consisting of one row of three
% and a second row of one, two or three).
% 
% The command \alignauthor (no curly braces needed) should
% precede each author name, affiliation/snail-mail address and
% e-mail address. Additionally, tag each line of
% affiliation/address with \affaddr, and tag the
% e-mail address with \email.
% 
% 1st. author
\alignauthor Manisha Verma\\
\affaddr{UCL Department Of Computer Science}\\
\email{\footnotesize manisha.verma.13@ucl.ac.uk}\\
%
\alignauthor Diego Ceccarelli\\
\affaddr{IMT Lucca}\\ 
\affaddr{ISTI CNR, Pisa}\\ 
\email{\footnotesize d.ceccarelli@isti.cnr.it}\\
% % 2nd. author
 \alignauthor Claudio Lucchese\\
 \affaddr{ISTI CNR, Pisa}\\ 
\email{\footnotesize c.lucchese@isti.cnr.it}\\
}



\maketitle
\begin{abstract}
With the creation and rapid development of knowledge bases, it has become easier
 to understand the underlying semantics of unstructured text (short or long) on the web.  
 In this work we especially look at the impact of entity linking on search logs. 
 Search queries are primary means of finding information. While a handful of 
 queries account for majority of search volume, a large number of queries 
 occur infrequently. The individual frequency and sparsity of related statistics makes 
 it difficult to draw any conclusions about such rare queries.
 %renders it difficult to draw any conclusions about the nature of the query. 
 In this work we focus on quantifying the extent of overlap between rare 
 (long tail) and popular (head) queries by means of entity linking. 
 We specifically analyze the frequency distribution of entities in head and 
 tail queries. 
 %We also analyze the ratio of popular entities to rare entities in tail queries. 
 Our analysis shows that by means of entity linking, we can indeed bridge the 
 gap between the head and tail. 

%Search is the core service of the Web. Everyday Google receives
%3.5 billion of queries. Being able to understand and summarize
%the needs that users express in their queries is a fundamental and interesting task. 
%In this paper we propose to use \emph{Entity Linking},
%a technique that allows to automatically link unstructured data 
%with entities in a Knowledge Base, for performing
%Web Usage Mining. In particular, we use Entity Linking for 
%exploring hidden connections between \emph{popular queries}, performed by huge number of users,  
%and \emph{rare queries}, usually performed by only one user. 



\end{abstract}
\category{H.3.3}{Information Storage and Retrieval}{Information Search and Retrieval}[Information Filtering, Search process]
%\vspace{-0.3cm}
%\terms{Algorithms, Design, Experimentation.}
%\vspace{-0.5cm}
\keywords{Entity linking; Annotations; Web Usage Mining.}


\section{Introduction}

At present, querying a search engine is the most popular means to find information. This makes
search log mining or analysis unavoidable as understanding user dynamics, demographics or interests
is instrumental in enriching user's search experience. \emph{Web Usage Mining}~\cite{silvestri2010mining}  
studies the user needs in order to improve the user satisfaction and engagement on the website. 

Search queries tend to follow a heay-tailed Zipf Distribution \cite{baeza2007impact}, wherein a large 
fraction of queries occur too infrequently. This large set is called the \emph{long tail}. While some 
popular queries \emph{head queries} are easy to analyze due to their tremendous search volume. 
But on the other hand, the small frequencies, corresponding low clicks or reformulations 
of tail queries makes it difficult to conclusively analyze them. Analysis of tail queries isnt new, 
however, existing work \cite{Doug2007Sigir,Goel2010Wsdm} looks at user behaviour associated with
rare queries whereas we believe that the content of such queries can also provide important cues in both
understanding as well as associating them with popular queries. 

But understanding such queries only by bag-of-words suffers from several shortcomings. However, with the 
creation and constant development of knowledge bases, its now possible to understand underlying semantics
of a query. Particulary, in this work, we propose to exploit \textbf{Entity Linking} (EL) for analyzing queries 
both at the head and the tail. Entity Linking helps enrich raw text with entities from a knowledge base; recently 
many approached for applying EL to queries have been proposed and evaluated during the Entity Recognition and Disambiguation Challenge\footnote{\url{http://web-ngram.research.microsoft.com/ERD2014/}} organized by Microsoft. 
%Everyday people perform millions of queries against search engines. Often different people express
%the same \emph{information need} using different terms due to their culture, previous experiences,
%age \dots. 
%Analysis of search logs to derive search patterns, user interests or demographics is instrumental 
%in enriching a user's experience with a search engine.   
%Being able to analyze this huge quantity of hetrogeneous data is a fundamental 
%task for a search engine, and in general for all the website owners that want to analyze queries 
%performed for reaching their website and to profile users. 


%\diego{describe what usually a query log contains?}
%A well known fact is the sparsity of the logs. In 2007, Yates \emph{ et al.}~\cite{baeza2007impact} 
%analyzed one year of logs and found that the $64\%$ of the queries were performed by only one user.
%Usually the frequency distribution of the queries follow the \emph{Zipf law}: there is a small 
%set of queries frequently searched by different users (the \emph{head}) and a large set of queries
%searched only by a few (or just one) users.

%While it is quite simple to analyze queries in the head, it is not so easy to manage the queries in the long tail, 
%due to the fact that they are huge in number and they are all different. 


Hollink \emph{et al.}~\cite{hollink2013web} exploited EL for web usage mining: 
in their work they consider a sample of queries related to entertainment from 
Yahoo! Search Logs. Their focus is to study the \emph{types} of the queries (e.g., trailer, movies, dvd), and on finding
type patterns among sessions and queries. In this work, we are interested in studying the 
relationship between the head and the tail queries through the entities they contain. 
Our primary research questions are:
\begin{itemize}
	\item Are queries in the tail just a different way to look for entities yet searched in the head? 
	\item Can we find tail queries about entities that are not searched in the head (we will call them \emph{tail entities})?
	\item Can we find a relationship between tail entities and \emph{head entities}?  
\end{itemize} 

The following section covers the technique used for annotating a large query log with entities and our 
preliminary findings on the enriched log.


 


%\manisha{observation:}

%\section{Related works}

%\input{related-works}

\section{Enriching AOL query log data}

We perform our analysis on the AOL query log~\cite{pass2006picture}, since it is easy to obtain\footnote{the query log can be retrieved from \url{http://www.gregsadetsky.com/aol-data/}}
AOL query log consists of approximately 20 millions of queries submitted by $650,000$ users from March to May in 2006. Queries are normalized (text lowercased, not ascii characters removed) and 
there are in total $10,154,742$ distinct queries. 

\begin{figure}
	\centering
	\includegraphics[scale=0.28]{images/aol}
	\caption{Query frequency distribution in AOL query log}
\end{figure}

\section{Analysis}
%
\begin{tabular}{ccccc}
\toprule
link probability & commonness & $k$ & $J^{spot}_k$ & $J^{top entity}_k$ \\
\midrule
\multirow{6}{*}{$0.3$} &  \multirow{3}{*}{$0.3$} & $50$ &  $0.01$  &   $0.03$  \\
					 &						   & $500$ 	&  $0.11$  &   $0.14$  \\
					 &						   & $5,000$ & $0.21$   &  $0.25$    \\
\cline{2-5}
					 &	\multirow{3}{*}{$0.7$}  & $50$      & $0.01$   &   $0.03$ \\
					 &						   & $500$   	& $0.11$   &   $0.14$  \\
					 &						   & $5,000$ 	& $0.21$   &   $0.27$    \\
					
\midrule		
\multirow{6}{*}{$0.7$} &  \multirow{3}{*}{$0.3$}   & $50$  & $0.05$   &   $0.06$  \\
					 &						   & $500$ 	   &   $0.11$   &    $0.13$  \\
					 &						   & $5,000$   &   $0.22$   &  $0.26$  \\
\cline{2-5}

					 &	\multirow{3}{*}{$0.7$}  & $50$   &  $0.05$   &  $0.06$  \\
					 &						   & $500$   &  $0.11$   &   $0.15$   \\
					 &						   & $5,000$ &  $0.22$   &   $0.28$  \\
\bottomrule
\end{tabular}

\begin{figure*}
	\footnotesize
	\centering
\begin{tabular}{lc|lc|lc|lc}
\toprule
\multicolumn{4}{c}{\head{}} & \multicolumn{4}{c}{\tail{}}\\
\multicolumn{2}{c}{\shead{}} & \multicolumn{2}{c}{\ehead{}} & \multicolumn{2}{c}{\stail{}} & \multicolumn{2}{c}{\etail{}}\\
\midrule
google         & 342,602  &  Google  		   & 349,337  &  florida 	 &	47,718	&	Florida 		& 49,366 \\
myspace        & 194,093  &  Yahoo\!  		   & 299,718  &  texas  	 &	 37,388  &   Texas   		& 37,526 \\
yahoo          & 142,361  &  Myspace 		   & 289,353  &  ohio    	 &	31,861   &   Ohio    		& 31,905 \\			
ebay           & 142,257  &   EBay   		   & 187,633  &  edu     	 &	26,641   &   New\_York        & 28,396 \\
yahoo.com      & 104,696  &  MapQuest          & 135,179  &  state   	 &	26,066   &   .edu    		& 26,642 \\
mapquest       & 88,617   &  Google\_Search     & 98,112   &  california  &   25,233  &   U.S.\_state      & 26,392 \\
google com     & 85,670   &  Hotmail           & 53,925   &  new york    &   24,865  &   California      & 25,859 \\
my space       & 48,401   &	  Bank\_of\_America  & 46,922   &  hotel   	 &	20,018   &   Real\_estate     & 25,232 \\
www.yahoo.com  & 44,198   &  Craigslist        & 45,586   &  real estate &   19,702  &   Myspace 		& 24,998 \\
internet       & 39,865   &  Ask.com           & 39,873   &  myspace 	 &	18,533   &   Floruit 		& 24,207 \\
ebay com       & 30,652   &  Internet          & 39,865   &  restaurant  &  17,065   &   Restaurant      & 21,996 \\
hotmail.com    & 28,492   &  Pornography       & 35,089   &  michigan    &   15,635  &   Hotel   		& 20,289 \\
map quest      & 27,949   &  Tattoo            & 33,113   &  new jersey  &   14,813  &   Nudity  		& 18,245 \\
craigslist     & 27,222   &  American\_Idol     & 28,890   &  georgia 	 &	14,525   &   United\_States   & 16,680 \\
american idol  & 23,665   &  Yahoo!\_Mail       & 28,238   &  black   	 &	13,921   &   Michigan        & 15,763 \\
\bottomrule
\end{tabular}
\end{figure*}



% 0.3	0.3	5000	7605	10000	-0.00425861172234	0.651603334904	0.251877816725
% 0.3	0.3	1000	7605	10000	0.000864864864865	0.967333790922	0.181334908447
% 0.3	0.3	500	7605	10000	0.0504689378758	0.0916275749584	0.142857142857
% 0.3	0.3	100	7605	10000	0.0707070707071	0.29725368045	0.0582010582011
% 0.3	0.3	50	7605	10000	0.0677551020408	0.487504639438	0.0309278350515

% 0.3	0.7	5000	6546	10000	0.000853770754151	0.927870523637	0.271617497457
% 0.3	0.7	1000	6546	10000	-0.00438838838839	0.835388377433	0.201923076923
% 0.3	0.7	500	6546	10000	0.0360240480962	0.228563309062	0.137656427759
% 0.3	0.7	100	6546	10000	0.0529292929293	0.435233981226	0.063829787234
% 0.3	0.7	50	6546	10000	0.0726530612245	0.456590780071	0.0309278350515

% 0.7	0.3	5000	6497	10000	0.00301676335267	0.749071345836	0.265182186235
% 0.7	0.3	1000	6497	10000	-0.0281201201201	0.18301667988	0.200480192077
% 0.7	0.3	500	6497	10000	-0.0581002004008	0.0521441951613	0.132502831257
% 0.7	0.3	100	6497	10000	0.0250505050505	0.71191420797	0.063829787234
% 0.7	0.3	50	6497	10000	0.00734693877551	0.939988975942	0.063829787234
%
% 0.7	0.7	5000	5787	10000	-0.00528201640328	0.57544778462	0.286835671085
% 0.7	0.7	1000	5787	10000	0.00147747747748	0.944225129062	0.223241590214
% 0.7	0.7	500	5787	10000	-0.0217394789579	0.467458493231	0.150747986191
% 0.7	0.7	100	5787	10000	-0.0379797979798	0.575557238742	0.0752688172043
% 0.7	0.7	50	5787	10000	-0.0220408163265	0.821317189868	0.063829787234


% 0.3	0.3	5000	9528	10000	0.0318708541708	0.000726887144919	0.205981669079
% 0.3	0.3	1000	9528	10000	-0.000996996996997	0.962346580915	0.137656427759
% 0.3	0.3	500	9528	10000	-0.028873747495	0.334506604282	0.108647450111
% 0.3	0.3	100	9528	10000	-0.0509090909091	0.45296240742	0.0582010582011
% 0.3	0.3	50	9528	10000	0.0367346938776	0.706605728077	0.010101010101
%
% 0.3	0.7	5000	9528	10000	0.0318708541708	0.000726887144919	0.205981669079
% 0.3	0.7	1000	9528	10000	-0.000996996996997	0.962346580915	0.137656427759
% 0.3	0.7	500	9528	10000	-0.028873747495	0.334506604282	0.108647450111
% 0.3	0.7	100	9528	10000	-0.0509090909091	0.45296240742	0.0582010582011
% 0.3	0.7	50	9528	10000	0.0367346938776	0.706605728077	0.010101010101
%
% 0.7	0.3	5000	7879	10000	-0.00860428085617	0.361609518155	0.221597849988
% 0.7	0.3	1000	7879	10000	0.0353993993994	0.0936976197984	0.156069364162
% 0.7	0.3	500	7879	10000	0.0364889779559	0.222613593133	0.108647450111
% 0.7	0.3	100	7879	10000	-0.0226262626263	0.738719854407	0.0582010582011
% 0.7	0.3	50	7879	10000	0.0563265306122	0.563821128257	0.0526315789474
%
% 0.7	0.7	5000	7879	10000	-0.00860428085617	0.361609518155	0.221597849988
% 0.7	0.7	1000	7879	10000	0.0353993993994	0.0936976197984	0.156069364162
% 0.7	0.7	500	7879	10000	0.0364889779559	0.222613593133	0.108647450111
% 0.7	0.7	100	7879	10000	-0.0226262626263	0.738719854407	0.0582010582011
% 0.7	0.7	50	7879	10000	0.0563265306122	0.563821128257	0.0526315789474

\subsection{Entity Stats}
Entities in Head ($E_{head}$): 8949
Entities in Tail ($E_{tail}$): 379342
Total Count of entities in Head : 6562784
Total Count of entities in Tail : 9764565
 

Tail queries have been often mapped to the head \cite{} by using bag-of-word representations.
Lately they have also been linked with the head queries through entities. However, in this work
we attempt to quantify the extent to which the entities in the tail can be linked to entities 
in the head queries. It is important to know how much will entity linking at tail benefit query mining. 
We shall, from now on refer to entities in head queries as `head entity' 
and entities occuring in the tail as `tail entity'. 

%1.no of queries with head and tail ent
%2.Head and tail entity histogram for queries with more than 1 ent 
We begin by analyzing the volume of entities found in the tail. Of X unique tail queries,
63.2\% contain atleast one entity. Of queries containing more than one entity,  
86.2\% queries contain atleast one head entity. Table \ref{table:entDist} shows the number of 
queries, corresponding number of entities and the head to tail entity ratio in query. 
Head to Tail entity ratio is calculated by dividing number of head entities with number of 
tail entities in the query. The table clearly shows that for every tail entity there is a 
head entity in the query. Although, there are queries with more than 6 entities, 
they are too few to draw any significant conclusions. 
\begin{table}
\caption{\#Entities vs \#Queries}
\label{table:entDist}
\centering
\begin{tabular}{|l|l|l|}
\hline
\#Entities & \#Queries & Head Tail Ratio \\ \hline
1 & 2622873 & NA \\ \hline
2 & 1731256 & 0.48 \\ \hline
3 & 457946  & 0.91 \\ \hline
4 & 74125 & 1.09  \\ \hline
5 & 10757 & 1.07 \\ \hline
6 & 2416  & 0.97 \\ \hline
\end{tabular}
\end{table}

Figure \ref{img:headTailEntBreakup} shows the number of queries for each ratio. 
WRITE MORE ABOUT THE PLOT

\begin{figure}[t]
\label{img:headTailEntBreakup}
\caption{\%Queries vs Head to Tail Entity Ratio}
  \centering
    \includegraphics[width = 0.45\textwidth]{images/entity-head-query-ratio-dist.png}
\end{figure}

%Head entity popularity
Given that there would several head entities, not all of them will be equally popular. 
Here, popularity of the entity refers to the scope of that entity, how many entities does 
it co-occur with, or how common is that entity. 
Figure \ref{img:headEntDist} shows the frequency distribution of head entities in the 
head queries. That is, how frequently does a head entity occur in popular queries. 
To aggregate statistics for the tail, we divide head entities into 7 buckets, each bucket
spanning 20\% of entities. This results in X popular entities being in 0-20\% range, Y 
entities in 20-40\% range, so on and so forth. 


\begin{figure}[t]
\label{img:headEntDist}
\caption{Cumulative Distribution of Entity Frequency in Head Queries}
  \centering
    \includegraphics[width = 0.45\textwidth]{images/entity-head-dist.png}
\end{figure}



The co-occurrance of tail entities with head entities of different grades is shown in 
Figure \ref{img:headEntDistInTail}. The figure clearly indicates that 
WRITE ABOUT THE PLOT HERE

 
\begin{table}
\caption{Queries per Popularity of Head Entity}
\label{table:headEntQueryDist}
\centering
\begin{tabular}{|l|l|l|}
\hline
Popularity & \#Entities & Queries \\ \hline
0-20\% & 3 & 14851 \\ \hline
20-40\% & 18  & 57611.0 \\ \hline
40-60\% & 95 & 149733.0 \\ \hline
60-80\% & 287 & 299561.0 \\ \hline
80-95\% & 560 & 569271.0 \\ \hline
95-97\% & 236 & 298498.0 \\ \hline
97-100\% & 349 & 854345. 0  \\ \hline
\end{tabular}
\end{table}


\begin{figure}[t]
\label{img:headRankInTail}
\caption{\%Queries vs Rank of Head Entities}
  \centering
    \includegraphics[width = 0.45\textwidth]{images/entity-head-query-ratio-dist.png}
\end{figure}

%In entity analysis of Aol query logs we calculate the following:
%\begin{itemize}
%\item No of queries with entities from the head or tail
%\item Frequency distribution of entities in head and tail
%\item Head and tail entity distribution in queries
%\end{itemize}





\newpage
\section{Future Roadmap}

We presented our preliminary results in using entity linking in order
to find hidden connections between popular and rare queries. 
In this work we observed that: i) spots and entities 
in the head and in the tail of the AOL query log follow similar trends but ii) 
the most frequent entities are different iii) the more spots a query contains 
(i.e., the longer it is) the highest is the probability that head entities and
tail entities co-occur. For future work, we would like to improve the quality 
of entity linking, implementing state of the art methods for entity linking on queries and
exploiting other informations in the logs (sessions, clicks\dots). Finally, it would 
be interesting studying the relationships among the entities mined in the querylog
(e.g., they occur in the same query, or in the same session, the share the same click \dots) and
compare them with the relations that we have in the knowledge base.

%\paragraph{\bf {Acknowledgements}}
%\section*{Acknowledgements}
%\vspace{-0.4cm}
\bibliographystyle{abbrv}
\bibliography{biblio}

\balancecolumns
% That's all folks!


\end{document}

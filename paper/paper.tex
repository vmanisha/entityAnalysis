% THIS IS SIGPROC-SP.TEX - VERSION 3.1
% WORKS WITH V3.2SP OF ACM_PROC_ARTICLE-SP.CLS
% APRIL 2009
%
% It is an example file showing how to use the 'acm_proc_article-sp.cls' V3.2SP
% LaTeX2e document class file for Conference Proceedings submissions.
% ----------------------------------------------------------------------------------------------------------------
% This .tex file (and associated .cls V3.2SP) *DOES NOT* produce:
%       1) The Permission Statement
%       2) The Conference (location) Info information
%       3) The Copyright Line with ACM data
%       4) Page numbering
% ---------------------------------------------------------------------------------------------------------------
% It is an example which *does* use the .bib file (from which the .bbl file
% is produced).
% REMEMBER HOWEVER: After having produced the .bbl file,
% and prior to final submission,
% you need to 'insert'  your .bbl file into your source .tex file so as to provide
% ONE 'self-contained' source file.
%
% Questions regarding SIGS should be sent to
% Adrienne Griscti ---> griscti@acm.org
%
% Questions/suggestions regarding the guidelines, .tex and .cls files, etc. to
% Gerald Murray ---> murray@hq.acm.org
%
% For tracking purposes - this is V3.1SP - APRIL 2009

%\documentclass{style/acm_proc_article-sp}

\documentclass{style/sig-alternate-2013}[9pt]

\usepackage{color}
%\usepackage{pgfplotstable}
\usepackage{booktabs}
\usepackage{array}
\usepackage{colortbl}
%\usepackage{algorithm2e}
%\usepackage{algorithm}
\usepackage{url}
\usepackage{stmaryrd}
\usepackage{enumitem}
\usepackage{xspace}
\usepackage{multirow}
%\usepackage{pgfplots}

%\usepackage{tikz}
%\usetikzlibrary{positioning}

\pdfpagewidth=8.5in
\pdfpageheight=11in

\definecolor{mybluei}{RGB}{124,156,205}
\definecolor{myblueii}{RGB}{73,121,193}
\definecolor{mygreen}{RGB}{202,217,126}

%\pgfdeclarelayer{background}
%\pgfsetlayers{background,main}


\newcommand{\comment}[2]{[\textbf{\color{red} #1}: \emph{#2}]\\}%
% uncomment below to hide comments
%\renewcommand{\comment}[2]{}

\newcommand{\diego}[1]{\comment{diego}{#1}}
\newcommand{\claudio}[1]{\comment{claudio}{#1}}
\newcommand{\manisha}[1]{\comment{manisha}{#1}}


\newcommand{\todo}[1]{\comment{todo}{#1}}
\newcommand{\tail}[1]{$Q_{tail}${}}
\newcommand{\head}[1]{$Q_{head}${}}
\newcommand{\stail}[1]{$S_{tail}${}}
\newcommand{\shead}[1]{$S_{head}${}}
\newcommand{\etail}[1]{$E_{tail}${}}
\newcommand{\ehead}[1]{$E_{head}${}}

\newcommand{\dexter}[1]{{\sf Dexter}\xspace}


\newcommand{\piccino}[1]{ {\footnotesize #1}}
\newcommand{\etal}[1]{ {\emph{et al.}\~{}}}
%\newcommand{\eg}[1]{ {\emph{e.g.,}\~{}}}


\newtheorem{example}{Example}[section]
\newtheorem{problem}{Problem}[section]
\newtheorem{proposition}{Proposition}[section]







\begin{document}
	
	
% --- Author Metadata here ---
\newfont{\mycrnotice}{ptmr8t at 7pt}
\newfont{\myconfname}{ptmri8t at 7pt}
\let\crnotice\mycrnotice%
\let\confname\myconfname%

\permission{Permission to make digital or hard copies of all or part of this work for personal or classroom use is granted without fee provided that copies are not made or distributed for profit or commercial advantage and that copies bear this notice and the full citation on the first page. Copyrights for components of this work owned by others than ACM must be honored. Abstracting with credit is permitted. To copy otherwise, or republish, to post on servers or to redistribute to lists, requires prior specific permission and/or a fee. Request permissions from permissions@acm.org.}
%\conferenceinfo{ESAIR'13,}{October 28, 2013, San Francisco, CA, USA.}
%\CopyrightYear{2013}
%\crdata{978-1-4503-2413-7/13/10 \\
%http://dx.doi.org/10.1145/2513204.2513212 }



\clubpenalty=10000 
\widowpenalty = 10000


\title{Mining User Needs in the Long Tail through Entity Linking}
%\subtitle{[Extended Abstract]
%\titlenote{A full version of this paper is available as
%\textit{Author's Guide to Preparing ACM SIG Proceedings Using
%\LaTeX$2_\epsilon$\ and BibTeX} at
%\texttt{www.acm.org/eaddress.htm}}}
%
% You need the command \numberofauthors to handle the 'placement
% and alignment' of the authors beneath the title.
%
% For aesthetic reasons, we recommend 'three authors at a time'
% i.e. three 'name/affiliation blocks' be placed beneath the title.
%
% NOTE: You are NOT restricted in how many 'rows' of
% "name/affiliations" may appear. We just ask that you restrict
% the number of 'columns' to three.
%
% Because of the available 'opening page real-estate'
% we ask you to refrain from putting more than six authors
% (two rows with three columns) beneath the article title.
% More than six makes the first-page appear very cluttered indeed.
%
% Use the \alignauthor commands to handle the names
% and affiliations for an 'aesthetic maximum' of six authors.
% Add names, affiliations, addresses for
% the seventh etc. author(s) as the argument for the
% \additionalauthors command.
% These 'additional authors' will be output/set for you
% without further effort on your part as the last section in
% the body of your article BEFORE References or any Appendices.

\numberofauthors{3} %  in this sample file, there are a *total*
% of EIGHT authors. SIX appear on the 'first-page' (for formatting
% reasons) and the remaining two appear in the \additionalauthors section.
%
\author{
% You can go ahead and credit any number of authors here,
% e.g. one 'row of three' or two rows (consisting of one row of three
% and a second row of one, two or three).
% 
% The command \alignauthor (no curly braces needed) should
% precede each author name, affiliation/snail-mail address and
% e-mail address. Additionally, tag each line of
% affiliation/address with \affaddr, and tag the
% e-mail address with \email.
% 
% 1st. author
\alignauthor Manisha Verma\\
\affaddr{UCL Department Of Computer Science}\\
\email{\footnotesize manisha.verma.13@ucl.ac.uk}\\
%
\alignauthor Diego Ceccarelli\\
\affaddr{IMT Lucca}\\ 
\affaddr{ISTI CNR, Pisa}\\ 
\email\footnotesize {d.ceccarelli@isti.cnr.it}\\
% % 2nd. author
 \alignauthor Claudio Lucchese\\
 \affaddr{ISTI CNR, Pisa}\\ 
\email{\footnotesize c.lucchese@isti.cnr.it}\\
}



\maketitle
\begin{abstract}
Search is the core service of the Web. Everyday Google receives
3.5 billion of queries. Being able to understand and summarize
the needs that users express in their queries is a fundamental and interesting task. 
In this paper we propose to use \emph{Entity Linking},
a technique that allows to automatically link unstructured data 
with entities in a Knowledge Base, for performing
Web Usage Mining. In particular, we use Entity Linking for 
exploring hidden connections between \emph{popular queries}, performed by huge number of users,  
and \emph{rare queries}, usually performed by only one user. 



\end{abstract}
\category{H.3.3}{Information Storage and Retrieval}{Information Search and Retrieval}[Information Filtering, Search process]
%\vspace{-0.3cm}
%\terms{Algorithms, Design, Experimentation.}
%\vspace{-0.5cm}
\keywords{Entity linking; Annotations; Web Usage Mining.}


\section{Introduction}




%\manisha{observation:}

%\section{Related works}

%


\section{Enriching AOL query log data}

We perform our analysis on the AOL query log, since it is publicly available\footnote{\url{http://www.gregsadetsky.com/aol-data/}}. AOL log consists of approximately 20 million queries submitted by $650,000$ users from March to May 2006. Queries are normalized (text lowercased, non ascii characters removed) and there are in total $10,154,742$ distinct queries. 
We extract 2 distinct sets from these queries: 
\begin{description}
	\item{\tail{}} The set of queries in the long tail, i.e. queries that appear in the log with a frequency \emph{lower than or equal} to $2$. The set contains $7,746,607$ distinct queries, i.e. $76\%$ of distinct queries, but it is $26\%$ of the total volume of the queries.
	\item{\head{}} The set of queries in the head. It contains queries that appear with a frequency \emph{greater than} $99$. The set contains $19,953$ distinct queries, i.e. $0.002\%$ if we look at the distinct queries, but still these queries represent $26\%$ of total query volume.% if we consider the frequencies of the queries.
\end{description}
Although, the two sets differ in number of queries ($\sim19$K versus $\sim7$M), they cover the same fraction of total queries issued to the search engine. All our analysis are performed on these two sets.

\paragraph{Enriching the Queries}
The first step of the analysis is to find entities in the search queries.  
%We are interested in studying the entities that may occur in queries in order to find connections between the queries in \head{} and in \tail{}.
The Entity Linking task consists of identifying small fragments of text (called \emph{spots}), which may refer to an entity (represented by a URI) within a knowledge base (KB). For example, `ny', `nyc' will link to New York City in a KB. 
Usually EL task consists of two steps: i) \textbf{Spot Detection}: given the input document (in our case a query), 
the spots are detected and for each spot a list of candidate entities is returned; and ii) \textbf{Disambiguation}:
for each ambiguous spot (e.g., \texttt{brazil} could refer to the country or the football team), a single entity is 
selected to be linked to the spot.

%In this preliminary work we decided to work with the queries and without considering the user sessions. 
We performed only the first step of the Entity Linking process: spot detection. 
We do not perform disambiguation for two reasons: i) Since we consider individual queries and not sessions, the context is not sufficient and probably not useful to correctly disambiguate the query, and ii) Disambiguation is usually computationally more expensive
since it involves the pairwise comparison of the candidate entities of the detected spots to compute \emph{relatedness}\cite{milne2008learning} distance between them.

We identify spots in both the head and the tail queries %in \tail{} and \head{} 
using Dexter~\cite{ceccarelli2013dexter}. The linker exploits a dictionary of 
more than 10 million spots extracted 
from titles and anchor texts of a recent English Wikipedia dump. Given a query, 
all possible $n$-grams (with $n$ between one and six, and considering only n-grams 
longer than 2 characters) are generated and matched against the dictionary. The system identifies 
at least one spot in $13,977$ ($70\%$) and $4,901,987$ ($63\%$) \head{} and \tail{} 
respectively. For each spot we also collect: 
i) the \textbf{position in the query}, the start and end position in the query ii)
 the \textbf{link probability}, the probability
of the spot linking to an entity, computed by the number of occurrences of the spot 
as anchor text in Wikipedia divided by the number of occurrences as plain text and iii)
 \textbf{the candidate entities}, a list of possible candidate entities for the spot; 
 for each candidate we also retrieve its \textbf{commonness}, the probability $p(e|s)$ that
  the spot $s$ refers to the entity $e$.



\section{Analysis}

\begin{tabular}{ccccc}
\toprule
link probability & commonness & $k$ & $J^{spot}_k$ & $J^{top entity}_k$ \\
\midrule
\multirow{6}{*}{$0.3$} &  \multirow{3}{*}{$0.3$} & $50$ &  $0.01$  &   $0.03$  \\
					 &						   & $500$ 	&  $0.11$  &   $0.14$  \\
					 &						   & $5,000$ & $0.21$   &  $0.25$    \\
\cline{2-5}
					 &	\multirow{3}{*}{$0.7$}  & $50$      & $0.01$   &   $0.03$ \\
					 &						   & $500$   	& $0.11$   &   $0.14$  \\
					 &						   & $5,000$ 	& $0.21$   &   $0.27$    \\
					
\midrule		
\multirow{6}{*}{$0.7$} &  \multirow{3}{*}{$0.3$}   & $50$  & $0.05$   &   $0.06$  \\
					 &						   & $500$ 	   &   $0.11$   &    $0.13$  \\
					 &						   & $5,000$   &   $0.22$   &  $0.26$  \\
\cline{2-5}

					 &	\multirow{3}{*}{$0.7$}  & $50$   &  $0.05$   &  $0.06$  \\
					 &						   & $500$   &  $0.11$   &   $0.15$   \\
					 &						   & $5,000$ &  $0.22$   &   $0.28$  \\
\bottomrule
\end{tabular}

\begin{figure*}
	\footnotesize
	\centering
\begin{tabular}{lc|lc|lc|lc}
\toprule
\multicolumn{4}{c}{\head{}} & \multicolumn{4}{c}{\tail{}}\\
\multicolumn{2}{c}{\shead{}} & \multicolumn{2}{c}{\ehead{}} & \multicolumn{2}{c}{\stail{}} & \multicolumn{2}{c}{\etail{}}\\
\midrule
google         & 342,602  &  Google  		   & 349,337  &  florida 	 &	47,718	&	Florida 		& 49,366 \\
myspace        & 194,093  &  Yahoo\!  		   & 299,718  &  texas  	 &	 37,388  &   Texas   		& 37,526 \\
yahoo          & 142,361  &  Myspace 		   & 289,353  &  ohio    	 &	31,861   &   Ohio    		& 31,905 \\			
ebay           & 142,257  &   EBay   		   & 187,633  &  edu     	 &	26,641   &   New\_York        & 28,396 \\
yahoo.com      & 104,696  &  MapQuest          & 135,179  &  state   	 &	26,066   &   .edu    		& 26,642 \\
mapquest       & 88,617   &  Google\_Search     & 98,112   &  california  &   25,233  &   U.S.\_state      & 26,392 \\
google com     & 85,670   &  Hotmail           & 53,925   &  new york    &   24,865  &   California      & 25,859 \\
my space       & 48,401   &	  Bank\_of\_America  & 46,922   &  hotel   	 &	20,018   &   Real\_estate     & 25,232 \\
www.yahoo.com  & 44,198   &  Craigslist        & 45,586   &  real estate &   19,702  &   Myspace 		& 24,998 \\
internet       & 39,865   &  Ask.com           & 39,873   &  myspace 	 &	18,533   &   Floruit 		& 24,207 \\
ebay com       & 30,652   &  Internet          & 39,865   &  restaurant  &  17,065   &   Restaurant      & 21,996 \\
hotmail.com    & 28,492   &  Pornography       & 35,089   &  michigan    &   15,635  &   Hotel   		& 20,289 \\
map quest      & 27,949   &  Tattoo            & 33,113   &  new jersey  &   14,813  &   Nudity  		& 18,245 \\
craigslist     & 27,222   &  American\_Idol     & 28,890   &  georgia 	 &	14,525   &   United\_States   & 16,680 \\
american idol  & 23,665   &  Yahoo!\_Mail       & 28,238   &  black   	 &	13,921   &   Michigan        & 15,763 \\
\bottomrule
\end{tabular}
\end{figure*}



% 0.3	0.3	5000	7605	10000	-0.00425861172234	0.651603334904	0.251877816725
% 0.3	0.3	1000	7605	10000	0.000864864864865	0.967333790922	0.181334908447
% 0.3	0.3	500	7605	10000	0.0504689378758	0.0916275749584	0.142857142857
% 0.3	0.3	100	7605	10000	0.0707070707071	0.29725368045	0.0582010582011
% 0.3	0.3	50	7605	10000	0.0677551020408	0.487504639438	0.0309278350515

% 0.3	0.7	5000	6546	10000	0.000853770754151	0.927870523637	0.271617497457
% 0.3	0.7	1000	6546	10000	-0.00438838838839	0.835388377433	0.201923076923
% 0.3	0.7	500	6546	10000	0.0360240480962	0.228563309062	0.137656427759
% 0.3	0.7	100	6546	10000	0.0529292929293	0.435233981226	0.063829787234
% 0.3	0.7	50	6546	10000	0.0726530612245	0.456590780071	0.0309278350515

% 0.7	0.3	5000	6497	10000	0.00301676335267	0.749071345836	0.265182186235
% 0.7	0.3	1000	6497	10000	-0.0281201201201	0.18301667988	0.200480192077
% 0.7	0.3	500	6497	10000	-0.0581002004008	0.0521441951613	0.132502831257
% 0.7	0.3	100	6497	10000	0.0250505050505	0.71191420797	0.063829787234
% 0.7	0.3	50	6497	10000	0.00734693877551	0.939988975942	0.063829787234
%
% 0.7	0.7	5000	5787	10000	-0.00528201640328	0.57544778462	0.286835671085
% 0.7	0.7	1000	5787	10000	0.00147747747748	0.944225129062	0.223241590214
% 0.7	0.7	500	5787	10000	-0.0217394789579	0.467458493231	0.150747986191
% 0.7	0.7	100	5787	10000	-0.0379797979798	0.575557238742	0.0752688172043
% 0.7	0.7	50	5787	10000	-0.0220408163265	0.821317189868	0.063829787234


% 0.3	0.3	5000	9528	10000	0.0318708541708	0.000726887144919	0.205981669079
% 0.3	0.3	1000	9528	10000	-0.000996996996997	0.962346580915	0.137656427759
% 0.3	0.3	500	9528	10000	-0.028873747495	0.334506604282	0.108647450111
% 0.3	0.3	100	9528	10000	-0.0509090909091	0.45296240742	0.0582010582011
% 0.3	0.3	50	9528	10000	0.0367346938776	0.706605728077	0.010101010101
%
% 0.3	0.7	5000	9528	10000	0.0318708541708	0.000726887144919	0.205981669079
% 0.3	0.7	1000	9528	10000	-0.000996996996997	0.962346580915	0.137656427759
% 0.3	0.7	500	9528	10000	-0.028873747495	0.334506604282	0.108647450111
% 0.3	0.7	100	9528	10000	-0.0509090909091	0.45296240742	0.0582010582011
% 0.3	0.7	50	9528	10000	0.0367346938776	0.706605728077	0.010101010101
%
% 0.7	0.3	5000	7879	10000	-0.00860428085617	0.361609518155	0.221597849988
% 0.7	0.3	1000	7879	10000	0.0353993993994	0.0936976197984	0.156069364162
% 0.7	0.3	500	7879	10000	0.0364889779559	0.222613593133	0.108647450111
% 0.7	0.3	100	7879	10000	-0.0226262626263	0.738719854407	0.0582010582011
% 0.7	0.3	50	7879	10000	0.0563265306122	0.563821128257	0.0526315789474
%
% 0.7	0.7	5000	7879	10000	-0.00860428085617	0.361609518155	0.221597849988
% 0.7	0.7	1000	7879	10000	0.0353993993994	0.0936976197984	0.156069364162
% 0.7	0.7	500	7879	10000	0.0364889779559	0.222613593133	0.108647450111
% 0.7	0.7	100	7879	10000	-0.0226262626263	0.738719854407	0.0582010582011
% 0.7	0.7	50	7879	10000	0.0563265306122	0.563821128257	0.0526315789474

Tail queries have been often mapped to the head \cite{} by using bag-of-word representations.
Lately they have also been linked with the head queries through entities. However, in this work
we attempt to quantify the extent to which the entities in the tail can be linked to entities 
in the head queries. It is important to know how much will entity linking at tail benefit query mining. 
We shall, from now on refer to entities in head queries as `head entity' 
and entities occuring in the tail as `tail entity'. 

%1.no of queries with head and tail ent
%2.Head and tail entity histogram for queries with more than 1 ent 
We begin by analyzing the volume of entities found in the tail. With X unique tail queries, we 
found that Y\% contain atleast one entity. Of queries containing more than one entity,  
Z\% queries contain atleast one head entity. Table \ref{table:entDist} shows total number of 
queries with number of entities. 

\begin{table}
\caption{#Entities vs #Queries}
\label{table:entDist}
\centering
\begin{tabular}{|l|l|}
\hline
#Entities & #Queries \\ \hline
1 & 2622873 \\hline
2 & 1731256 \\ \hline
3 & 457946 \\ \hline
4 & 74125 \\ \hline
5 & 10757 \\ \hline
6 & 2416 \\ \hline
\end{tabular}
\end{table}

Figure \ref{img:headTailEntBreakup} shows the percentage of head and tail entities with number 
of entities in a query. Although, there are queries with more than 6 entities, 
they are too few to draw any significant conclusions. 

%
Given that there would several head entities, there is a pattern in the 


In entity analysis of Aol query logs we calculate the following:
\begin{itemize}
\item No of queries with entities from the head or tail
\item Frequency distribution of entities in head and tail
\item Head and tail entity distribution in queries
\ent{itemize}






\section{Future Roadmap}

We presented our preliminary results on using entity linking 
to find hidden connections between popular and rare queries. 
In this work we observed that: i) spots and entities 
in the head and in the tail of the AOL query log follow similar trends but ii) 
the most frequent entities are different iii) the more spots a query contains 
(i.e., the longer it is) the higher is the probability that head entities and
tail entities co-occur. For future work, we would like to improve the quality 
of entity linking, implementing state of the art methods on queries and
exploiting other informations in the logs (sessions and reformulations). 
Finally, it would be interesting to study the relationships among the entities in the query log
%(e.g., they occur in the same query, or in the same session, the share the same click \dots) 
and compare them with the relations present in the knowledge base.

%\paragraph{\bf {Acknowledgements}}
%\section*{Acknowledgements}
%\vspace{-0.4cm}
\bibliographystyle{abbrv}
\bibliography{biblio}

\balancecolumns
% That's all folks!


\end{document}

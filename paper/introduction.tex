
Everyday people perform millions of queries against search engines. Often different people express
the same \emph{information need} using different terms due to their culture, previous experiences,
age \dots. 
Analysis of search logs to derive search patterns, user interests or demographics is instrumental 
in enriching a user's experience with a search engine.   
\emph{Web Usage Mining}~\cite{silvestri2010mining}  
studies the user needs in order to improve the user satisfaction and engagement on the website. 
%Being able to analyze this huge quantity of hetrogeneous data is a fundamental 
%task for a search engine, and in general for all the website owners that want to analyze queries 
%performed for reaching their website and to profile users. 

\diego{describe what usually a query log contains?}
A well known fact is the sparsity of the logs. In 2007, Yates \emph{ et al.}~\cite{baeza2007impact} 
analyzed one year of logs and found that the $64\%$ of the queries were performed by only one user.
Usually the frequency distribution of the queries follow the \emph{Zipf law}: there is a small 
set of queries frequently searched by different users (the \emph{head}) and a large set of queries
searched only by a few (or just one) users. This large set is called the \emph{long tail}.

While it is quite simple to analyze queries in the head, it is not so easy to manage the queries in the long tail, 
due to the fact that they are huge in number and they are all different. In this work, we propose
to exploit \textbf{Entity Linking} (EL) for analyzing the logs. Entity Linking helps enrich raw text with entities 
from a knowledge base; recently many approached for applying 
EL to queries have been proposed and evaluated during the Entity Recognition and Disambiguation Challenge\footnote{
\url{http://web-ngram.research.microsoft.com/ERD2014/}} organized by Microsoft. 

Hollink \emph{et al.}~\cite{hollink2013web} exploited EL for web usage mining: in their inspiring work they consider a sample of 
queries from Yahoo! Search in the United States, and they filter only queries about movies. The focus
of their work is on studying the \emph{types} of the queries (e.g., trailer, movies, dvd), and on finding
type patterns among sessions and queries. In our research we are interested in studying the 
relationship between the head and the tail of a query log
through the entities potentially captured in the queries. Our primary research questions are:
\begin{itemize}
	\item Are queries in the tail just a different way to express information needs of popular queries? 
	\item Can we find tail queries that express needs that are not in the head (let us call them \emph{tail needs})?
	\item Can we find a relationship between tail needs and \emph{head needs}?  
\end{itemize} 

In the following we will describe the technique that we adopted for annotating a large query log with entities 
and our preliminary findings on the enriched log.


 

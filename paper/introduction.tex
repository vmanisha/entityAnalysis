
Everyday people perform millions of queries against search engines. Often different people formulate
the the same \emph{information need} in different ways based on their culture, previous experiences,
age \dots. Being able to analyze this huge quantity of inhomogeneous data is a fundamental 
task for search engine, and in general for all the website owners that can analyze queries 
performed for reaching their website. \emph{Web Usage Mining}\diego{add citations} allows 
to study the user needs in order to improve the user satisfaction an engagement on the website. 

\diego{describe what usually a query log contains?}
A well known fact is the sparsity of the sparsity of the logs. In 2007, Yates \etal\cite{baeza2007impact} 
analyzed one year of logs and found that the $64\%$ of the queries were performed by only one user.
Usually the frequency distribution of the queries follow the \emph{Zipf law}: there is a small 
set of queries frequently searched by different users (the \emph{head}) and a large set of queries
searched only by a few (or just one) users. This is large set is called the \emph{long tail}.
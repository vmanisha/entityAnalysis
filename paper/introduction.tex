At present, querying a search engine is the most popular means to find information. This makes
search log mining or analysis unavoidable as understanding user dynamics, demographics or interests
is instrumental in enriching user's search experience. \emph{Web Usage Mining}~\cite{silvestri2010mining}  
studies the user needs in order to improve the user satisfaction and engagement on the website. 

Search queries tend to follow a heavy-tailed Zipf Distribution \cite{baeza2007impact}, wherein a large 
fraction of queries occur too infrequently. This large set is called the \emph{long tail}. While some 
popular queries \emph{head queries} are easy to analyze due to their tremendous search volume. 
But on the other hand, the small frequencies, corresponding low clicks or reformulations 
of tail queries makes it difficult to conclusively analyze them. Analysis of tail queries is not new, 
however, existing work \cite{Doug2007Sigir,Goel2010Wsdm} looks at user behavior associated with
rare queries whereas we believe that the content of such queries can also provide important cues in both
understanding as well as associating them with popular queries. 

But understanding such queries only by bag-of-words suffers from several shortcomings. However, with the 
creation and constant development of knowledge bases, it is now possible to understand underlying semantics
of a query. Particularly, in this work, we propose to exploit \textbf{Entity Linking} (EL) for analyzing queries 
both at the head and the tail. Entity Linking helps enrich raw text with entities from a knowledge base; recently 
many approached for applying EL to queries have been proposed and evaluated during the Entity Recognition and Disambiguation Challenge\footnote{\url{http://web-ngram.research.microsoft.com/ERD2014/}} organized by Microsoft. 
%Everyday people perform millions of queries against search engines. Often different people express
%the same \emph{information need} using different terms due to their culture, previous experiences,
%age \dots. 
%Analysis of search logs to derive search patterns, user interests or demographics is instrumental 
%in enriching a user's experience with a search engine.   
%Being able to analyze this huge quantity of hetrogeneous data is a fundamental 
%task for a search engine, and in general for all the website owners that want to analyze queries 
%performed for reaching their website and to profile users. 


%\diego{describe what usually a query log contains?}
%A well known fact is the sparsity of the logs. In 2007, Yates \emph{ et al.}~\cite{baeza2007impact} 
%analyzed one year of logs and found that the $64\%$ of the queries were performed by only one user.
%Usually the frequency distribution of the queries follow the \emph{Zipf law}: there is a small 
%set of queries frequently searched by different users (the \emph{head}) and a large set of queries
%searched only by a few (or just one) users.

%While it is quite simple to analyze queries in the head, it is not so easy to manage the queries in the long tail, 
%due to the fact that they are huge in number and they are all different. 


Hollink \emph{et al.}~\cite{hollink2013web} exploited EL for web usage mining: 
in their work they consider a sample of queries related to entertainment from 
Yahoo! Search Logs. Their focus is to study the \emph{types} of the queries (e.g., trailer, movies, dvd), and on finding
type patterns among sessions and queries. In this work, we are interested in studying the 
relationship between the head and the tail queries through the entities they contain. 
Our primary research questions are:
\begin{itemize}
	\item Are queries in the tail just a different way to look for entities yet searched in the head? 
	\item Can we find tail queries about entities that are not searched in the head (we will call them \emph{tail entities})?
	\item Can we find a relationship between tail entities and \emph{head entities}?  
\end{itemize} 

The following section covers the technique used for annotating a large query log with entities and our 
preliminary findings on the enriched log.


 
